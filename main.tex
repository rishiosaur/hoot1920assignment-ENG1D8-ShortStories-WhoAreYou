\documentclass[]{article}
\usepackage{graphicx}	
\usepackage{amssymb}
\usepackage{amsmath}
\usepackage{subfiles}
\usepackage{geometry}
\usepackage{braket}
\usepackage{hyperref}

\usepackage{todonotes}


\hypersetup{
    colorlinks=true,
    linkcolor=black,
    filecolor=cyan,      
    urlcolor=cyan,
    pdftitle={Phi Design},
}

\usepackage{listings}
\usepackage{xcolor}

\usepackage{arev}

\usepackage{fontspec}
    \setmainfont{SF Pro Text Regular}
    \setmonofont{Fira Mono}

\usepackage{titlesec}
\titleformat{\chapter}[display]
  {\normalfont\huge}
  {\bfseries\chaptertitlename\ \thechapter}{20pt}{\Huge}

\titleformat{\section}
    {\normalfont\Large}
    {\thesection}{1em}{}

\titleformat{\subsection}
    {\normalfont\itshape\large}
    {\thesubsection}{1em}{}

\definecolor{codegreen}{rgb}{0.3,0.3,0.3}
\definecolor{codegray}{rgb}{0.5,0.5,0.5}
\definecolor{codepurple}{rgb}{0.69,0.69,0.69}
 
\lstdefinestyle{mystyle}{ 
    commentstyle=\color{codegreen},
    keywordstyle=\color{codegreen}\normalfont,
    numberstyle=\tiny\color{codegray},
    stringstyle=\color{codepurple},
    basicstyle=\ttfamily\footnotesize,
    breakatwhitespace=false,         
    breaklines=true,                 
    captionpos=b,                    
    keepspaces=true,                 
    numbers=left,                    
    numbersep=5pt,   
	tabsize=2,
    basicstyle=\ttfamily
}
 
\lstset{style=mystyle}


\usepackage{titling}
\renewcommand\maketitlehooka{\null\mbox{}\vfill}
\renewcommand\maketitlehookd{\vfill\null}

\title{\Huge Assignment: \textbf{Get to know you}}
\author{\LARGE Rishi Kothari}
\date{ENG1D8}

\geometry{margin=1in}

\begin{document}
\pagenumbering{gobble}
\maketitle

\newpage

\pagenumbering{arabic}

\section*{Part 1}
\subsection*{Planning}
\subsubsection*{Personality and Interests}
\begin{itemize}
    \item Programming
    \begin{itemize}
        \item Been programming and designing for 6 years
        \item Programming has taught me to think in creative but logical ways
    \end{itemize}
    \item I think that I'm 
\end{itemize}
\subsubsection*{Goals in School}
\begin{itemize}
    \item In past years, my goals for school have been to get great marks, and \textit{just} get great marks.
    \item However, this made me prize my self-worth on marks
    \item This led to me having a \textit{very} toxic attitude towards my friends and the people in my life.
    \item Eventually, I realized that marks were temporary, in a sense.
    \item They will always fluctuate, and there isn't really a point in putting your self-worth on test marks.
    \item This year, I'm putting a focus on the connections that I'll make in school, rather than focusing exclusively on my marks.
    \item This means that I have a focus on sociality in this semester.
\end{itemize}
\subsubsection*{Someone I'd like to meet}
\begin{itemize}
    \item I'd like to meet Leibniz
    \item He was one of the greatest polymaths of the 17th and 18 centuries
    \item This era was also called the "Enlightenment era", simply because of the amount of maths and science that were going on at that time.
    \item Expertise ranged from maths to philosophy
    \begin{itemize}
        \item One of his greatest achievements was creating the foundations of calculus
        \item This is one of my favorite units in math, and I would love to talk about item
        \item He was also a theologist, and I'd love to see what his thoughts were about religion at that time.
    \end{itemize}
    \item I'm by no means a good mathematician, but I enjoy doing maths, and Leibniz was one of the best in his field.
\end{itemize}

\subsection*{Paragraphs}
I'm Rishi, a programmer, wannabe mathematician, and someone who values connection. I've been programming for about 6 years, and in that time, it's given me a couple of things: my logical attitude towards problems and a sense of confidence, because I can do anything with it! Programming has also pushed me to do better in academics; it's a 'nerdy' thing to do, which isn't bad -scoring high is always good!- However, I soon prioritized my academics over \textit{everything} else in my life to keep getting those marks, and near the end of Grade 8, when my marks started to drop, I paid the price, both socially and mentally. That's why in high school, I'm putting my connections with people as a priority and goal in school, so that I become a well-rounded student, rather than somebody who only cares about marks. However, programming isn't the \textit{only} part of my life. I also love math, and I'd love to discuss it with someone that loves math beyond anybody else in history: Leibniz.

Leibniz is known as one of the greatest 17th-century polymaths, and he is the person that I feel best aligns with my beliefs. He is best known for his work in the governments of old, consulting about relatively obscure topics ranging from math to philosophy to theology, which is why I'd love to converse with him. 
He was a pioneer of my favorite unit in math -calculus-, and helped develop the modern idea of philosophy. These are both ideas that have shaken the foundations of thinking, and thinkers along with it. As a result, I feel that both of us would be able to take something away from our conversation: He could go back to the 17th century with a very strong idea of what the future holds, and I would be able to better understand function analysis and what thought truly means to the 17\textup{th} century mind.


\section*{Part 2}
\subsection*{Planning}
\subsubsection*{My word}
Discursive
\subsubsection*{My symbol}
<Picture of wheel>

\begin{itemize}
    \item Discursive mainly means to cover a wide range of subjects, and is mostly used in a negative way; straying off-topic and going onto 
\end{itemize}

The symbol I've chosen is the humble wheel, and is representative of innovation and symmetry. The wheel is constantly overlooked, even if it has solved some of the biggest problems in our civilization to date. What this means to me is that innovation can come from anywhere: nobody planned on creating the wheel, but lo and behold: it exists! I've taken this philosophy to my heart; in my daily life, I innovate. For instance, in a math test, answers to hard questions can come from anywhere, and just like the wheel, nobody planned it. I describe myself as being "discursive", which simply means "to cover a wide range of subjects". The wheel is also symbolic of symmetry, and well-roundedness; it's a circle. This is representative of my aim as a student: not to become a wheel, but to become a well-rounded student! This also ties into my word: "Discursive", which means to cover a wide range of subjects. I think that I have knowledge on specific subjects, but being well-rounded means having a skill set that is both usefully wide \textit{and} deep.

\end{document}  